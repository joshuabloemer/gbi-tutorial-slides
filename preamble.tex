\usepackage[utf8]{inputenc}
\usepackage[ngerman]{babel}
\usepackage{amsmath,amssymb}
\usepackage{tikz}
\usepackage{calc}
% \usepackage{hyperref}
\usepackage{advdate}
\usepackage{csquotes}
\usepackage{mathtools}
\usepackage{siunitx}
\usepackage{ulem}
\usepackage{datetime2}
\usepackage[ruled]{algorithm2e}
\usepackage{xcolor}
\usepackage[T1]{fontenc}
\usepackage{pifont}
\usepackage{mathrsfs}
\usepackage{latexsym}
\usepackage{bm}
\usepackage{pgfplots}
\usepackage{booktabs}
\usepackage{tabularx}
\usepackage{graphics}
\usepackage{multirow}
\usepackage{geometry}

\DTMnewdatestyle{currdate}{%
  \renewcommand*\DTMdisplaydate[4]{##3.##2.##1}%
  \renewcommand*\DTMDisplaydate{\DTMdisplaydate}%
}
\DTMsetdatestyle{currdate}

\makeatletter
\appto\FixDate{%
  \edef\@dtm@currentyear{\the\year}%
  \edef\@dtm@currentmonth{\the\month}%
  \edef\@dtm@currentday{\the\day}%
}
\makeatother

\renewcommand{\vec}[1]{\mathaccent"017E {#1}}

\newcounter{tutweek}
\newcommand{\affiliation}{}



% commands %%%%%%%%%%%%%%%
\newcommand{\eulere}{\mathrm{e}}
\renewcommand{\emph}[1]{\textbf{\color{green1}#1}}
\renewcommand{\cite}[2]{{\color{green1}[\textbf{#1}; #2]}}
\newcommand{\EX}[1]{\mathbb E\left[ #1 \right]}
\newcommand{\Pro}[1]{\mathrm{Pr}\left[ #1 \right]}
\newcommand{\dif}[1]{\mathop{}\!\mathrm{d}#1\mathop{}}
\DeclareMathOperator{\len}{len}
% pseudocode %%%%%%%%%%%%%
\newcommand{\class}[1]{\keyword{class} \textbf{\color{gray1}\textsc{#1}}}
% \newcommand{\function}[1]{\textbf{\color{blue1}#1}}
\newcommand{\keyword}[1]{\textbf{\color{green1}#1}}
\newcommand{\type}[1]{\textit{\color{gray1}#1}}
\newcommand{\comment}[1]{{\color{red2}// #1}}
% settings 
\german
\renewcommand{\vec}[1]{\mathaccent&quot;017E {#1}}
\newcommand{\qCite}[1]{\smash{\textcolor{KITblue}{\raisebox{0.1ex}{[}{\small #1}\raisebox{0.1ex}{]}}}\vphantom{Ig}}
\newcommand{\alert}[1]{\textcolor{KITgreen}{{\bf #1}}}
\newcommand{\eg}[1]{\textcolor{KITblack!40}{(#1)}}
\newcommand{\rk}[1]{\smash{\raisebox{0.2ex}{(}{#1}\raisebox{0.2ex}{)}}}
\newcommand{\ek}[1]{\smash{\raisebox{0.15ex}{[}{#1}\raisebox{0.15ex}{]}}}
\newcommand{\mm}[1]{\scalebox{1.12}{\textrm{\textit{#1}}}}
\newcommand{\ml}{\raisebox{0.1ex}{-}}
\newcommand{\mx}{\raisebox{0.1ex}{+}}

% Farben
\definecolor{kit-green}{RGB}{0, 150, 130}
\definecolor{kit-blue}{RGB}{70, 100, 170}
\definecolor{kit-gray70}{rgb}{0.3, 0.3, 0.3}
\definecolor{kit-gray30}{rgb}{0.7, 0.7, 0.7}
\definecolor{kit-orange}{RGB}{223, 155, 27}

\newcommand{\cmark}{\ding{51}}%
\newcommand{\xmark}{\ding{55}}%
\newcommand{\impl}{\ensuremath{\Rightarrow}}
\newcommand{\appimpl}{\ensuremath{\rightsquigarrow}}
\newcommand{\eps}{\ensuremath{\varepsilon}}
\newcommand{\gdw}{\ensuremath{\Leftrightarrow}}
\newcommand{\assign}{\ensuremath{\gets}}
\newcommand{\card}[1]{\lvert #1 \rvert}
\newcommand{\olim}{ \displaystyle{\lim_{n \to \infty}}}
\newcommand{\ceil}[1]{\lceil {#1} \rceil}
\newcommand{\floor}[1]{\lfloor {#1} \rfloor}
\newcommand{\seq}[1]{\langle {#1} \rangle}
\newcommand{\range}[2]{\seq{{#1}, \dots, {#2}}}
\newcommand{\inc}{\texttt{++}}
\newcommand{\word}[1]{\textcolor{blue}{\texttt{#1}}}

% O-Notation
\renewcommand{\O}[1]{\ensuremath{\mathcal{O}}(#1)}
\newcommand{\Th}[1]{\Theta\!\left(#1\right)}
\newcommand{\Oh}[1]{\O\!\left(#1\right)}
\newcommand{\Om}[1]{\Omega\!\left(#1\right)}
\newcommand{\oeq}{\asymp}
\newcommand{\ogeq}{\succeq}
\newcommand{\oleq}{\preceq}

% geklaut aus der VL
\DeclareMathOperator{\dist}{dist}
% pseudocode %%%%%%%%%%%%%
% \newcommand{\class}[1]{\keyword{class} \textbf{\color{gray1}\textsc{#1}}}
% \newcommand{\function}[1]{\textbf{\color{blue1}#1}}
%\newcommand{\keyword}[1]{\textbf{\color{green1}#1}}
% \newcommand{\type}[1]{\textit{\color{gray1}#1}}
%\newcommand{\comment}[1]{{\color{red2}// #1}}

\newcommand{\ifNotEmpty}[2]{\if\relax\detokenize{#1}\relax\else#2\fi}

% geklaut von karina-folien
\newcommand{\alg}[1]{{\bfseries\sffamily\scshape\color{kit-blue}#1}}
\newcommand{\method}[1]{\texttt{#1}}
\newcommand{\datatype}[1]{{\color{kit-gray70}\textsl{\textsf{#1}}}}
\newcommand{\kwfont}[1]{\textbf{\sffamily\textcolor{kit-green}{#1}}}
\newcommand{\func}[1]{\textbf{\sffamily{\textcolor{kit-blue}{#1}}}}
\newcommand{\SetDatatype}[1]{\expandafter\newcommand\csname #1\endcsname{\datatype{#1}}}
% change bar of algorithm2e
\makeatletter
\renewcommand{\algocf@Vsline}[1]{%    no vskip in between boxes but a strut to separate them,
  \strut\par\nointerlineskip% then interblock space stay the same whatever is inside it
  \algocf@push{\skiprule}%        move to the right before the vertical rule
  \hbox{{\color{kit-gray30}\vrule width 2pt}%               the vertical rule
    \vtop{\algocf@push{\skiptext}%move the right after the rule
      \vtop{\algocf@addskiptotal\advance\hsize by -\skiplength #1}}}% inside the block
  \algocf@pop{\skiprule}}% restore indentation
\makeatother

\SetKwSty{kwfont}
\SetFuncSty{func}
\SetFuncArgSty{textsf}
\SetArgSty{textsf}
\SetKwIF{If}{ElseIf}{Else}{if}{then}{else if}{else}{}
\SetKwFor{While}{while}{do}{}
\SetKwRepeat{Do}{do}{while}
\SetKwFor{For}{for}{do}{}
\SetKwProg{Fn}{}{}{}
\SetKw{continue}{continue}
\SetKwFunction{size}{size}
\SetKwFunction{ehmpty}{empty} % This is not a typo, it is supposed to be ehmpty, else things break
\SetKwFunction{enqueue}{enqueue}
\SetKwFunction{dequeue}{dequeue}
\SetKwFunction{N}{N}
\SetKwFunction{push}{push}
\SetKwFunction{pushBack}{pushBack}
\SetKwFunction{popMin}{popMin}
\SetKwFunction{decPrio}{decPrio}
\SetKwFunction{DFS}{DFS}
\DontPrintSemicolon
% usage: \begin{algo}[arguments]{name} <actual algorithm> \end{algo}
\newenvironment{algo}[2][]{
    \NoCaptionOfAlgo
    \SetAlFnt{\sffamily}
    \begin{algorithm}[H]
        \caption{\alg{#2}\ifNotEmpty{#1}{(#1)}:}
        \SetKwFunction{#2}{#2}
}{
    \end{algorithm}
}

% usage: \quickAlgo{name}{args}{<actual algorithm>}
\newcommand{\quickAlgo}[3]{
  \Fn{\func{#1}(#2):}{#3}
}
